\documentclass[11pt]{article}

\usepackage{amsmath}
\usepackage{amssymb}
\usepackage{centernot}
\usepackage{mathtools}

%define some new macros
\newcommand{\qed}{\\ \begin{flushright} // \end{flushright}}
\DeclarePairedDelimiter\ceil{\lceil}{\rceil}
\DeclarePairedDelimiter\floor{\lfloor}{\rfloor}

\setlength{\topmargin}{-.5in}
\setlength{\textheight}{9in}
\setlength{\oddsidemargin}{.125in}
\setlength{\textwidth}{6.25in}

\begin{document}

\title{Dummit \& Foote Abstract Algebra 3rd Ed}

\maketitle
\clearpage

\section*{Chapter 0}
	\subsection*{0.1.1}
		This exercise is contained within 0.1.4
	
	\subsection*{0.1.2}
		$(Q + P)X = QX + PX = XQ + XP = X(Q + P)$ Thus $(Q + P) \in B$
	\subsection*{0.1.3}
		$(QP)X = Q(PX) = Q(XP) = (QX)P = (XQ)P = X(QP)$ Thus $(QP) \in B$
	\subsection*{0.1.4}
		Take $p,q,r,s \in \Re$ s.t.\\
		\\
		$
		A = \begin{pmatrix}
			p & q \\
			r & s
		\end{pmatrix}
		$ and $
		AX = XA \implies
		\begin{pmatrix}
			p & p+q \\
			r & r+s
		\end{pmatrix} =
		\begin{pmatrix}
			p+r & q+s \\
			r & s
		\end{pmatrix}
		$\\
		\\
		From here, we compare the entries on either side; we see $p = p+r \implies r = 0$,\\ and $p+q=q+s \implies p=s$\\
		Thus the general form for $A \in B$ is $
		\begin{pmatrix}
			a & b \\
			0 & a
		\end{pmatrix}
		$\\
		As a sanity check, we check $AX = XA$ and get $
		\begin{pmatrix}
			a & a+b\\
			0 & a
		\end{pmatrix} = 
		\begin{pmatrix}
			a & a+b \\
			0 & a
		\end{pmatrix}
		$
	\subsection*{0.1.5}
		$(a)$ No, take $f(1/2) = 1$ and $f(2/4) = 2$, thus $ a = a \centernot \implies f(a) = f(a)$\\
		$(b)$ Yes, since there is no information lost in this map, it must be well defined (i.e. you aren't throwing away any piece of the input)\\
	\subsection*{0.1.6}
		This is a well defined map; each real number has a unique decimal representation, thus there is no way to change the first digit after the decimal point.\\
	\clearpage
	\subsection*{0.1.7}
		This relation is predicated on the = relation under the image of $f$, so this is clearly a equivalence relation, but we will show the properties nonetheless:\\
		\indent Reflexive - $a \sim a \implies f(a) = f(a) \ \checkmark$ \\
		\indent Symmetric - $a \sim b \implies f(a) = f(b) \implies f(b) = f(a) \implies b \sim a \ \checkmark$ \\
		\indent Transitive - $a \sim b,\  b \sim c \implies f(a) = f(b), \ f(b) = f(c) \implies f(a) = f(c) \implies a \sim c\ \checkmark$ \\
		This relation is the definition of a fiber, as it relates all elements of the domain with the same value under $f$.  If $f$ were not surjective,
		we could find some element $b \in B$ such that $f(a) \centernot = b \ \forall a \in A$, and the fiber of $f$ over $b$ is the empty set. 
		This empty set breaks our equivalence partitioning for our relation; however if we restrict $f$ to surjection, the relation partitions $A$ nicely into equivalence classes!
	\subsection*{0.2.1}
		\indent Syntax: $ax + by = gcd; lcm = (xy)/gcd$\\
		$(a)$ $2*20 + (-3)*13 = 1; lcm = (20*13)$\\
		$(b)$ $27*69 + (-5)*372  = 3; lcm = (23*372)$\\
		$(c)$ $8*792 + (-23)*275 = 11; lcm = (792*25)$\\
		$(d)$ $(-126)*11391 + 253*5673 = 3; lcm = (3797*5673)$\\
		$(e)$ $(-105)*1761 + 118*1567 = 1; lcm = (1761*1567)$\\
		$(f)$ $(-17)*507885 + 142*60808 = 691; lcm = (735*60808)$
	\subsection*{0.2.2}
		We have for $a, b, n, m \in \mathbb{Z}; a = nk;\ b = mk \implies as + bt = (nk)s + (mk)t = k(ns + mt)$ and $k$ divides $as + bt \ \forall s,t \in \mathbb{Z}$
	\subsection*{0.2.3}
		We have $n = mk$ for some $m, k \in \mathbb{Z}$.  Take $a = mq$ and $b = kp$ where $k \centernot | q$ and $m \centernot | p$, thus $n \centernot | a$ and
		$n \centernot | b$.  Consider $ab = mqkp = (mk)qp = n(qp) \implies n | ab$
	\subsection*{0.2.4}
		$ax + by = a(x_0 + \frac{b}{d}t) + b(y_0 - \frac{a}{d}t) = ax_0 + \frac{ab}{d}t + by_0 - \frac{ab}{d}t = ax_0 + by_0 + (\frac{ab}{d} - \frac{ab}{d})t = ax_0 + by_0 + 			0t$; this is clearly invariant under choice of t and represents a valid solution space.
	\subsection*{0.2.5}
		$\varphi(1)=1;\ \varphi(2)=1;\ \varphi(3)=2;\ \varphi(4)=2;\ \varphi(5)=4;\\ \varphi(6)=2;\ \varphi(7)=6;\ \varphi(8)=4;\ \varphi(9)=6;\ \varphi(10)=4;$	\\
		$\varphi(11)=10;\ \varphi(12)=4;\ \varphi(13)=12;\ \varphi(14)=6;\ \varphi(15)=8;\\ \varphi(16)=8;\ \varphi(17)=16;\ \varphi(18)=6;\ \varphi(19)=18;\ \varphi(20)=8;$	\\
		$\varphi(21)=12;\ \varphi(22)=10;\ \varphi(23)=22;\ \varphi(24)=8;\ \varphi(25)=20;\\ \varphi(26)=12;\ \varphi(27)=18;\ \varphi(28)=12;\ \varphi(29)=28;\ \varphi(30)=8;$
	\subsection*{0.2.6}
		Take $S \subset \mathbb{N}$ and $P$ to be the complement of $S$ with $1 \in S$ and $s \in S \implies s+1 \in S$\\
		Now take $p \in P$ such that $p$ is the minimal element in $P$.  We know $p \centernot = 1$ since $1 \in S$\\
		Thus $p - 1$ exists and can't be in $P$ since $p$ is the minimal element of $P$.  $p -  1 \centernot \in P \implies
		p - 1 \in S \implies p - 1 + 1 = p \in S$.  From here we see $p$ is in $S$ and the complement of $S$ and can not exist,
		Thus $P$ must be empty and $S = \mathbb{N}$
	\subsection*{0.2.7}
		The power of $p$ in $pb^2$ is bound to be odd, where the power of $p$ in $a^2$ is bound to be even.\\ 
		More explicitly,taking $a, b \in \mathbb{Z}$ we can write $a = k_1^{a_1} \dots k_n^{a_n}p^{a_p}; b = q_1^{b_1} \dots q_n^{b_n}p^{b_p}$ where $q_i, k_i$ are primes. 
		This means $a^2 = k_1^{2a_1} \dots k_n^{2a_n}p^{2a_p}$ and $pb^2 = q_1^{2b_1} \dots q_n^{2b_n}p^{2b_p + 1}$, so we need to find $a_p, b_p$ s.t.
		$2a_p = 2b_p + 1$ though this is impossible.
		
	\subsection*{0.2.8}
		First we start by counting up to $n$ by multiples of $p$. Note there are $\floor*{\frac{n}{p}}$ such numbers.  At this point, we have counted up all single multiples of
		$p$, though we have yet to account for the multiples of $p^2$ (i.e. every $p^{th}$ multiple of $p$).  In order to get the number of $p^2$ terms, we count up to 
		$n$ over multiples of $p^2$ for a total number of $\floor*{\frac{n}{p^2}}$ (looks familiar).  This counting method continues up to the $ith$ power.  Now to arrive at the
		largest power of $p$ that divides into $n$, we sum up all these terms: $\sum\limits_{i \in \mathbb{N}}\floor*{\frac{n}{p^i}}$
	\subsection*{0.2.9}
		Haskell implementation\\
		$linearGCD :: Integer -> Integer -> (Integer, Integer, Integer)\\
		linearGCD\ a\ b = (d, u, v)\ where\\
		    \indent (d, x, y) = eGCD\ 0\ 1\ 1\ 0\ (abs\ a)\ (abs\ b)\\
		    \indent u\ |\ a < 0     = negate\ x\\
		    \indent\ \ \   |\ otherwise = x\\
		    \indent v\ |\ b < 0     = negate\ y\\
		    \indent\ \ \  |\ otherwise = y\\
		    \indent eGCD\ n1\ o1\ n2\ o2\ r\ s\\
		    \indent\ \ \  | (s == 0)    = (r, o1, o2)\\
		    \indent\ \ \  |\ otherwise = case\ r\ `quotRem`\ s\ of\\
		    \indent \indent                 (q, t) -> eGCD\ (o1 - q*n1)\ n1\ (o2 - q*n2)\ n2\ s\ t$\\
	\clearpage
	\subsection*{0.2.10}
		Let $p$ be a prime larger than $N+1$ such that $\varphi(p^k) = (p-1)(p^{k-1}) > N$.  Therefore any prime $q$ dividing $n$ is no larger than $N+1$ and there are only
		finitely many choices for this $q$. Furthermore, we know $\varphi(n) = \varphi(q^k)\varphi(m)$ for some number $m$ that is not divisible by $q$.  Note that $\varphi(m)$ is 
		constant, so this equation relies on $k$.  This limits $k \leq log_q(\frac{N}{m})$.  Now we see that both the choice for $q$ and $k$ are of a finite set, thus there are finitely
		many numbers such that $\varphi(n) = N$.\\ 
		\\
		Let's say there is some number $M$ such that $\varphi(n) < M \forall n \in \mathbb{N}$.  Since we are mapping an infinite set, $\mathbb{N}$, onto a finite set, we are bound to
		break the finite ceiling we set in the last previous portion (i.e. at least one of the numbers from $1\dots M$ will have an infinite amount of numbers mapped to it).  Thus we
		may not incur a maximum $M$.
	\subsection*{0.2.11}
		Take $d = p_1^{a_1} \dots p_n^{a_n}$ where $p_i$ is a prime that divides d.  Since $d|n$, $n = p_1^{b_1} \dots p_n^{b_n}q$ where $a_i \leq b_i$.\\
	  	From here, $\varphi(d) = \varphi(p_1^{a_1}) \dots \varphi(p_n^{a_n}) = p_1^{a_1 - 1} \dots p_n^{a_n - 1}(p_1 - 1) \dots (p_n - 1)$ and \\ 
		$\varphi(n) = \varphi(p_1^{b_1})\varphi(p_2^{b_2}) \dots \varphi(p_n^{b_n})\varphi(q) = p_1^{b_1 - 1} \dots p_n^{b_n - 1}(p_1 - 1) \dots (p_n - 1) \varphi(q)$\\
		Now since $a_i - 1 \leq b_i - 1$, $\varphi(d)|\varphi(n)$

	\subsection*{0.3.1}
		$\mathbb{Z}/18\mathbb{Z} =$\{$\overline{0},\overline{1},\overline{2},\overline{3},\overline{4},\overline{5},\overline{6},\overline{7},\overline{8},\overline{9},
		\overline{10},\overline{11},\overline{12},\overline{13},\overline{14},\overline{15},\overline{16},\overline{17}$\}\\
		Where $\overline a = \{ x \in \mathbb{Z} | x = 18k + a  \}$
	\subsection*{0.3.2}
		For some $a \in \mathbb{Z}$, we have $a = qn + r \implies a \equiv r\ (mod\ n) \implies \overline{a} = \overline{r}$  Since $ 0\leq r < n$, $\overline{a} \in \mathbb{Z}/n\mathbb{Z}$; if we take $0 \leq a,b < n$ where $a \centernot = b$ and $a-b > 0 \implies n \centernot | a-b \implies \overline{a} \centernot = \overline{b}$ thus the residue classes are distinct.
	\subsection*{0.3.3}
		First, note that $10 \equiv 1\ (mod\ 9)$ so $10^n \equiv 1^n\ (mod\ 9)$.\\
		  Now we have $a \equiv \overline{a_n10^{n} + a_{n-1}10^{n-1} + \dots + a_0} \equiv \overline{a_n10^n} + \dots + \overline{a_0} \equiv \overline{a_n}\overline{10^n} + \dots + \overline{a_0} \equiv \overline{a_n} + \dots + \overline{a_0}\ (mod\ 9)$
	\subsection*{0.3.4}
		$37 \equiv 8\ (mod\ 29); 8^2 \equiv 6\ (mod\ 29); 8^4 \equiv 7\ (mod\ 29); 8^8 \equiv 20\ (mod\ 29);$\\
		$ 8^{16} \equiv 23\ (mod\ 29); 8^{32} \equiv 7\ (mod\ 29); 8^{64} \equiv 20\ (mod\ 29)$\\
		$8^{100} = 8^{64}8^{32}8^4 \equiv 20*7*7\ (mod\ 29) \equiv 23\ (mod\ 29)$
	\subsection*{0.3.5}
		$9^5 \equiv 49\ (mod\ 100); 9^{10} \equiv 1\ (mod\ 100); (9^{10})^{150} \equiv (1)^{150} \equiv 1\ (mod\ 100)$
	\subsection*{0.3.6}
		$0*0 \equiv 0\ (mod\ 4);1*1 \equiv 1\ (mod\ 4); 2*2 \equiv 0\ (mod\ 4); 3*3 \equiv 1\ (mod\ 4)$
	\subsection*{0.3.7}
		As seen in the previous exercise, the addition of any two squares can only equal 0, 1, and 2.
	\subsection*{0.3.8}
		We can see right away the only way $a^2, b^2, c^2 \in \mathbb{Z}/4\mathbb{Z}$  can satisfy this is if $a^2 = b^2 = c^2 = 0$.  This means $a, b, c$ are all even.  We now see $a^2, b^2, c^2$ all have a factor of $2^2$ which implies there is a smaller solution available.
	\subsection*{0.3.9}
		This problem reduces to showing that the odd elements of $\mathbb{Z}/8\mathbb{Z}$ square to $\overline{1}$\\
		$1*1 \equiv 1\ (mod\ 8); 3*3 \equiv 1\ (mod\ 8); 5*5 \equiv 1\ (mod\ 8); 7*7 \equiv 1\ (mod\ 8)$ 
	\subsection*{0.3.10}
		The elements in $(\mathbb{Z}/n\mathbb{Z})^*$ are all numbers with multiplicative inverses modulo n.  In order for $a^{-1}$ to exist, $(a, n) = 1$. By definition, $\varphi(n)$ is the number of relatively prime numbers less than $n$.  This will be hashed out in more detail in the following exercises!
	\subsection*{0.3.11}
		Take $a, b \in (\mathbb{Z}/n\mathbb{Z})^* \implies \exists a^{-1}, b^{-1}$ such that $a*b*b^{-1}*a^{-1} = a*1*a^{-1} =  a*a^{-1} = 1$\\
		Thus $a*b \in (\mathbb{Z}/n\mathbb{Z})^*$ where $b^{-1}*a^{-1}$ is the inverse
	\subsection*{0.3.12}
		We can write $a = a_1*(a,n); n = n_1*(a, n); a*n_1 = a_1*n_1*(a, n) = a_1*n \equiv 0\ (mod\ n)$.  Now assume $\exists c\ s.t.\ ac \equiv 1\ (mod\ n) \implies \exists k\ s.t.\ kn = ac - 1$  Since $(a, n) | ac - kn\ \forall c, k \in \mathbb{Z}/n\mathbb{Z} \implies (a, n) | 1 \implies (a, n) = 1$.  However, we have taken $(a, n) > 1$ and a contradiction arises!
	\subsection*{0.3.13}
		From Euclid's algorithm, we have $ac + kn = (a, n) = 1$ for some $c, k \in \mathbb{Z}/n\mathbb{Z} \implies ac \equiv 1\ (mod\ n)$
	\subsection*{0.3.14}
		From 12 we see all relatively prime $a$ can't have a multiplicative inverse in $\mathbb{Z}$, and from 13 we see all relatively prime $a$ has an inverse than may be computed with Euclid's algorithm; thus we have $(\mathbb{Z}/n\mathbb{Z})^* = {a | (a, n) = 1}$  For a concrete example, we can provide elements that either send some $a$ to 0, or 1\\
	$1*1 \equiv 1\ (mod\ 12); 2*6 \equiv 0\ (mod\ 12); 3*4 \equiv 0\ (mod\ 12); 5*5 \equiv 1\ (mod\ 12);\\ 7*7 \equiv 1\ (mod\ 12); 8*3 \equiv 0\ (mod\ 12); 9*4 \equiv 0\ (mod\ 12); 10*6 \equiv 0\ (mod\ 12); 11*11 \equiv 1\ (mod\ 12)$
	\subsection*{0.3.15}
		(a) $(13)^{-1} \equiv 17\ (mod\ 20)$\\
		(b) $(69)^{-1} \equiv 40\ (mod\ 89)$ \\
		(c) $(1891)^{-1} \equiv 253\ (mod\ 3797)$ \\
		(d) $(6003722857)^{-1} \equiv 77695236753 \ (mod\ 77695236973)$ \\
	\subsection*{0.3.16}
		Haskell implementation\\
		$reduceMod :: Integer -> Integer\\
		reduceMod\ a\ n\ |\ (a < 0)\ =\ reduceMod\ (a + n)\ n\\
		\indent \indent \indent \indent\ \ \ |\ (a > n -1)\ =\ reduceMod\ (a - n)\ n\\
		\indent \indent \indent \indent\ \ \ |\ otherwise\ =\ a\\ \\
		multiMod :: Integer -> Integer -> Integer -> Integer\\
		multiMod\ a\ b\ n\ =\ reduceMod\ (a*b)\ n\\ \\ 
		addMod :: Integer -> Integer -> Integer -> Integer\\
		addMod\ a\ b\ n\ =\ reduceMod\ (a + b)\ n\\ \\
		getInverse :: Integer -> Integer -> Maybe\ Integer\\
		getInverse\ a\ n\ =\ relativePrime\  where \\
		\indent \indent \indent \indent \indent \ \ \ relativePrime\ |\ (gcd\ ==\ 1)\ =\ Just\ (reduceMod\ inverse\ n)\\
		\indent \indent \indent \indent \indent \indent \indent \indent \indent \indent |\ otherwise\ =\ Nothing\\
		\indent \indent \indent \indent \indent \indent \indent \indent \indent \indent (gcd,\ inverse, \_)\ =\ linearGCD\ a\ n$

\section*{Chapter 1}
	\subsection*{1.1.1}
	\subsection*{1.1.2}
	\subsection*{1.1.3}
	\subsection*{1.1.4}
	\subsection*{1.1.5}
	\subsection*{1.1.6}
	\subsection*{1.1.7}
	\subsection*{1.1.8}
	\subsection*{1.1.9}
	\subsection*{1.1.10}
	\subsection*{1.1.11}
	\subsection*{1.1.12}
	\subsection*{1.1.13}
	\subsection*{1.1.14}
	\subsection*{1.1.15}
	\subsection*{1.1.16}
	\subsection*{1.1.17}
	\subsection*{1.1.18}
	\subsection*{1.1.19}
	\subsection*{1.1.20}
	\subsection*{1.1.21}
	\subsection*{1.1.22}
	\subsection*{1.1.23}
	\subsection*{1.1.24}
	\subsection*{1.1.25}
	\subsection*{1.1.26}
	\subsection*{1.1.27}
	\subsection*{1.1.28}
	\subsection*{1.1.29}
	\subsection*{1.1.30}
	\subsection*{1.1.31}
	\subsection*{1.1.32}
	\subsection*{1.1.33}
	\subsection*{1.1.34}
	\subsection*{1.1.35}
	\subsection*{1.1.36}
	\subsection*{1.2.1}
	\subsection*{1.2.2}
	\subsection*{1.2.3}
	\subsection*{1.2.4}
	\subsection*{1.2.5}
	\subsection*{1.2.6}
	\subsection*{1.2.7}
	\subsection*{1.2.8}
	\subsection*{1.2.9}
	\subsection*{1.2.10}
	\subsection*{1.2.11}
	\subsection*{1.2.12}
	\subsection*{1.2.13}
	\subsection*{1.2.14}
	\subsection*{1.2.15}
	\subsection*{1.2.16}
	\subsection*{1.2.17}
	\subsection*{1.2.18}


		
\end{document}