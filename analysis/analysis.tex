\documentclass[11pt]{article}

\usepackage{amsmath}
\usepackage{amssymb}
\usepackage{centernot}
\usepackage{mathtools}

%define some new macros
\newcommand{\qed}{\\ \begin{flushright} // \end{flushright}}
\DeclarePairedDelimiter\ceil{\lceil}{\rceil}
\DeclarePairedDelimiter\floor{\lfloor}{\rfloor}

\setlength{\topmargin}{-.5in}
\setlength{\textheight}{9in}
\setlength{\oddsidemargin}{.125in}
\setlength{\textwidth}{6.25in}

\begin{document}

\title{Baby Rudin 3rd Ed}

\maketitle
\clearpage

\section*{Chapter 1}
	\subsection*{1.1}
		(+): Take $ r = \frac{a}{b}$ so $ r + x = \frac{a}{b} + x = \frac{y}{z} \implies x = \frac{y}{z} - \frac{a}{b} =  \frac{by - az}{bz}$ and $x$ must be rational\\
		(*): Take $r = \frac{a}{b}$ so $rx = \frac{xa}{b} = \frac{y}{z} \implies x = \frac{by}{az}$ and $x$ must be rational
	\subsection*{1.2}
		Notice this problem reduces to proving the irrationality of $\sqrt{3}$\\
		Take $a, b$ to be coprime integers, $b \centernot = 0 \implies \frac{a^2}{b^2}  = 3 \implies a^2 = 3b^2$  $a, b$ must be odd, since they would not be
		coprime otherwise.  Now take $a = 2i + 1; b = 2n + 1; i \centernot = n \implies (2i + 1)^2 = 3(2n + 1)^2 \implies 4i^2 + 4i + 1 = 12n^2 + 12n + 3 
		\implies 2(i^2 + i) = 2(3n^2 + 3n) + 1$  Since the $i$ terms describe an even integer, and the $n$ terms describe an odd integer, it is impossible to choose
		$i, n$ such that $a^2 = 3b^2$
	\subsection*{1.3}
		(a) $xy = xz \implies y = 1*y = (x^{-1}x)y = (x^{-1})xy = (x^{-1})xz = (x^{-1}x)z = 1*z = z$\\
		(b) As above, take $z = 1$\\
		(c) As above, take $z = x^{-1}$\\
		(d) From (c), $x^{-1}x = 1 \implies x = (x^{-1})^{-1}$
	\subsection*{1.4}
		$\forall x \in E,\  \alpha \leq x \leq \beta \implies \alpha \leq \beta$ 
	\subsection*{1.5}
	\subsection*{1.6}
	\subsection*{1.7}
	\subsection*{1.8}
	\subsection*{1.9}
	\subsection*{1.10}
	\subsection*{1.11}
	\subsection*{1.12}
	\subsection*{1.13}
	\subsection*{1.14}
	\subsection*{1.15}
	\subsection*{1.16}
	\subsection*{1.17}
	\subsection*{1.18}
	\subsection*{1.19}
	\subsection*{1.20}

\end{document}